\chapter{Summary}

This dissertation proposes a method of extracting review-like tweets from Twitter and then incorporating the sentiment of those tweets into a recommender system. This research aims to evaluate to what extent Twitter can provide a suitable source of online reviews that can be used effectively in the generation of recommendations in a recommender system. The project focused on reviews about hotels in Dublin.

Online reviews have become an important source of consumer information. They provide vast amounts of valuable information on consumer preferences. Twitter is a huge microblogging social media site with 126 million users and 500 million tweets posted daily. Twitter users regularly share their opinions on a wide range of different topics. Often these tweets can take the form of a review. This project aims to utilise these review-like tweets that are posted to Twitter.

Previous literature has shown that supervised machine learning methods can be successfully used to classify tweets. Numerous studies have focused particularly on classifying the sentiment of tweets. However, there are few papers which have looked at classifying tweets as reviews, which is what this research aims to do. One of the objectives of this research is to investigate if it is possible to classify whether a tweet contains content that could be deemed to be a review.

The data was collected through the Twitter Streaming API (application programming interface). It then had to be filtered so that it only contained tweets about hotels posted from Dublin. Once the dataset was filtered it had to be manually annotated so that it could be used to train a classification algorithm. A webpage was built to facilitate this annotation process. The tweets were labelled as 'Review', 'Some Content' or 'Irrelevant'.

Thirteen different classifiers were evaluated. These included: the Decision Tree Classifier, the Random Forest Classifier, the Multi Layer Perceptron Classifier, the Support Vector Machine Classifier, the Logistic Regression Classifier, the K Nearest Neighbours Classifier, the Gaussian Process Classifier, the AdaBoost Classifier, the Gaussian Naive Bayes Classifier, the Bernoulli Naive Bayes Classifier, the Multinomial Naive Bayes Classifier, the Quadratic Discriminant Analysis Classifier and the Linear Discriminant Analysis Classifier. Seven feature representations were experimented with, unigram bag-of-words, unigram, bigram and trigram TFIDF, unigram TFIDF with stop words removed, Word2Vec and Doc2Vec. The classifiers were implemented using Python's Scikit Learn library.

The results show that the Support Vector Machine Classifier performed best at classifying the tweets that contained reviews. It achieved a precision score of 74\%, a recall score of 74\%, an f1-score of 73\% and an accuracy score of 74.4\%. This showed that text classification is a valid method of extracting reviews from Twitter.

The next part of the project involved using the tweets, now classified as reviews, to help generate recommendations in a recommender system. The increasing amount of information available on the internet has meant that recommender systems have grown in importance. They are required to help users quickly find relevant information and to overcome the information overload problem. Hotel recommender systems such as Expedia, TripAdvisor and Booking.com are widely used by consumers.

It was shown in previous research that incorporating reviews from TripAdvisor into a hotel recommender system increased the average user satisfaction. In this project, reviews from Twitter were incorporated into the CoRE recommender system, in the hope that it would perform equally well.

The CoRE recommender system is a Cold Start Resistant and Extensible Recommender system. It makes use of collaborative filtering, content-based recommendation and contextual suggestion. The algorithm is designed to work in cold start situations and to be easily extensible.

Sentiment analysis was carried out on the tweets that were classified as reviews, using the Stanford NLP Sentiment Analyser. The sentiment scores produced were used to re-rank the results of the CoRE Recommender System and produce SentiCoRE.

The results show that incorporating the sentiment score had the desired effect and adjusted the rankings of the hotels. However, in terms of mean percentile rank (MPR) SentiCoRE performed worse that CoRE. 
