\chapter{Introduction}

The rapid development and expansion of the internet has introduced a new way for individuals to express their opinions. Online reviews have become a hugely important source of information for consumers. They play an important role in determining whether a person is satisfied with a product or service. Online reviews provide a huge amount of data on consumer preferences. These days it is extremely unlikely that someone will purchase a product, reserve a table at a restaurant or book a room in a hotel without first taking a look at some online reviews. These reviews inform and influence consumer decisions, and have a direct relationship with online sales. Several studies explore the relationship between online reviews and sales. Q. Ye, R. Law, B. Gu, and W. Chen found that a 10\% increase in a hotel's user rating can boost online bookings by over 5\% \cite{HotelSales2011}, suggesting that online reviews have a significant impact on online bookings. I. E. Vermeulen and D. J. T. m. Seegers' \cite{Vermeulen2009} research concluded that online reviews improve the average probability for consumers to consider booking a room in the reviewed hotel. 

This project will focus on Twitter, a microblogging social media site, where users can post short blocks of text of no more that 280 characters. Twitter currently has 126 million monetizable daily active users, with over 500 million tweets posted per day \cite{Twitter2019}. Tweets can often take the form of a 'review'. Twitter users post tweets regularly, expressing their ideas and opinions on a wide array of topics. An individual may for example tweet about, a city they visited, a hotel they stayed in, a restaurant they ate at or a film they watched. These review-like tweets can give us insight into consumers’s opinions on the entities that they interact with. With millions of tweets being posted every day, Twitter can be viewed as a huge source of underutilised reviews.

Traditional online review sites include websites like Tripadvisor, Foursquare and Yelp. Often these sites encourage users to leave reviews of hotels, restaurants or products. This method of obtaining reviews can result in more forced, manufactured reviews. In general, people tend to post more spontaneously and frequently to Twitter. As a result, the reviews on Twitter tend to contain more honest and unfiltered opinions. 

The language used in tweets is very different to a lot of longer form text and needs to be treated differently and needs to be treated differently. Tweets can be very informal, using casual language and slang. They contain things like hashtags, emoticons, twitter handles, URLs, images, videos and gifs.

\section{Research Question}
This research project will investigate whether Twitter can be used as an alternative or additional source of reviews for use in a recommender system. We will explore methods of classifying review-like tweets, identifying the review's sentiment and the effect of applying this information in a recommender system.

Recommender systems provide suggestions for products or services that are most likely to be of interest to a particular user \cite{Ricci2015}. They recommend based on users behaviours and preferences, such as explicit user ratings and reviews of products, and implicit user clicks and view times.

The following research question will be addressed:
\begin{center}
\textbf{Can twitter provide a suitable alternative to traditional online review sites?}
\end{center}

\section{Research Objectives}
The project will focus on tweets about hotels in the Dublin area. A collection of 2.5 million tweets posted from Dublin, collected between October 2017 and September 2018, will be used as the dataset. The main objective of the project is to determine whether tweets can be used successfully as a source of reviews in a recommender system.

The objectives of this research project are:
\begin{enumerate}
    \item \textbf{Data Processing.}\newline
    The raw dataset will be filtered first so that it only contains tweets posted from Dublin, and secondly so that it only contains tweets about hotels.
    \item \textbf{Dataset Annotation} \newline
    This part of the project will involve building a annotated dataset of tweets. The tweets from the processed dataset will be manually labelled as either review-like tweets, tweets that contain some content or irrelevant tweets.
    \item \textbf{Tweet Classification.}\newline
    The annotated set of tweets will be used to build a classifier. Various supervised machine learning techniques will be explored.
    \item \textbf{Sentiment Analysis.}\newline
    The Stanford NLP Sentiment Analyzer \cite{stanfordSentiment2013} will be used for sentiment analysis. A sentiment score will be generated for each tweet.
    \item \textbf{Recommender System}\newline
    The sentiment score produced will be applied to a pre-existing recommender system. The affect of adding the sentiment score will be analysed.
\end{enumerate}

\section{Report Structure}
The dissertation will be structured as follows; Chapter 2 presents a review of relevant literature. Chapter 3 describes... Chapter 4 ... Finally Chapter 7 presents our conclusions.