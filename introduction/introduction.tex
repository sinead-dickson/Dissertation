\chapter{Introduction}
Online reviews have become an important source of information for consumers. These days it is unlikely that someone will purchase a product, reserve a table at a restaurant or book a room in a hotel without first looking at the online reviews.

Twitter currently has 326 million active users, with over 500 million tweets posted per day. Tweets often take the form of a 'review'. Someone may for example tweet about, a city they visited, a hotel they stayed in, a restaurant they ate at or a film they watched. These review like tweets can give us insight into consumers’s opinions on the entities that they interact with. With millions of tweets being posted every day, Twitter can be looked at as a huge source of unutilized reviews.

Traditional sources of online reviews include sites like Tripadvisor, Foursquare or Yelp. Often these sites prompt users to review a hotel or restaurant after their stay. This can result in more forced reviews. In general, people tend to be more spontaneous in what they post to Twitter. As a result, the reviews contain more honest unfiltered opinions. 

\section{Research Question}
The aim of this research project is to investigate whether twitter can be used as an alternative or additional source of reviews a recommender system. We will explore methods of classifying review-like tweets, identifying the review's sentiment and using this information in a recommender system.

The following research question will be addressed:

\textbf{Can twitter provide a suitable alternative to traditional online review sites?}\\

\section{Context of Use/Motivation}
Added to a recommender system.

\section{Goal/Objective}
The objective of this research project is to develop a system to detect review like tweets and to perform sentiment analysis on these reviews. 

This project will focus on detecting and analysing reviews about hotels in the Dublin area. A collection of tweets posted from Dublin, collected between October 2017 and present, will be used as the dataset.
The main objectives of the project are:
• To identify entities (hotels) in tweets.
• To determine if the tweet is reviewing the entity.
• To analyse the sentiment towards that entity.

The data set consists of a collection of ~2.5 million tweets from Dublin. The tweets were collected using the public twitter API. The API allows you to specify the bounding box you want the tweets to have been posted from. The data set contains some tweets from outside the Dublin area. Pre-processing will involve filtering out all those tweets.

There are two approaches to identifying review like tweets that will be explored:
• The first approach is to begin by focusing on identifying entities in the tweets, and then determine whether the tweets are reviews.
• The second approach is to start by determining whether a tweet is a review, and then check if it mentions an entity.

\section{Report Structure}
The dissertation will be structured as follows:
Chapter 2 presents a review of relevant literature. Chapter 3 describes... Chapter 4 ... Finally Chapter 7 presents our conclusions.

\section{Sentiment Analysis}

\section{Machine Learning}

Throughout the document, a sans-serif font should be used. The font size should be 12 pt for main text. The text should be left justified and without hyphenation. Avoid italics and boldface in the main text. These font requirements comply with TCD policy on accessibility.

The page number should appear at the bottom of each page starting at 1 on the first page of the Introduction chapter. 

\section{Headings of sections and subsections}
Chapters should be divided into appropriate subsections. The section should be numbered sequentially from 1 within each chapter (e.g. 1.1, 1.2, 1.3 etc.).

\subsection{Subsection name style}
The subsections if used should be numbered sequentially within each section. You should really try to avoid using subsubsections, but if you do they should not be numbered.

\subsubsection*{Subsection}
This is a subsection. The asterisk after the command means it won't be printed with a number.

\subsection{Length of the report}
The page margins are set to 2.54 cm top, bottom, left and right. There may be a table or figure for which it is sensible to deviate from these margins, but in general the main text should be formatted within the specified margins.

The body of the report should be organized into several chapters. There are a number of chapters that you must have: an introduction; a background or literature review chapter; and a conclusion chapter. The focus of the other chapters will depend on your specific project.

The body of the report must be no more 60 pages for MAI. This does not include the front matter, references list and any appendices. In other words, from the first page of the Introduction to the last page of the Conclusions chapters must be less than 60 pages for MAI.

If you exceed these page limits or deviate significantly from this format, you will lose marks.

\section{Contents of the Introduction}
The introduction presents the nature of the problem under consideration, the context of the problem to the wider field and the scope of the project. The objectives of the project should be clearly stated.

\section{Contents of the background chapter}
The second chapter is typically a literature review, or survey of the state of the art, or a detailed assessment of the context and background for the project. The exact nature of this chapter depends on the topic and/or methods of the project. It is essential that the work of other people is properly cited. This will be discussed in detail in chapter 2 below. Note that you should use references wherever is appropriate through the report, not just in the literature review chapter.

\section{The Conclusions chapter}
The final chapter should give a short summary of the key methods, results and findings in your project. You should also briefly identify what, if any, future work might be executed to resolve unanswered questions or to advance the study beyond the scope that you identified in Chapter 1.
